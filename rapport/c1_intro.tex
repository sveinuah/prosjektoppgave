%===================================== CHAP 1 =================================

\chapter{Introduction}

As computational power of computers increase, and the processing and graphical units get smaller in size, there are more more focus on using cameras for control applications. Even though GPS based systems can calculate positions down to centimeter or even millimeter precision\cite{GPSaccuracy}, they will be less accurate in areas with large buildings, indoors, in tunnels and similar areas. This is because the GPS signal requires free line of sight to the tracked target to properly calculate its position.

Using cameras instead of, or in addition to, GPS is a good alternative. Using Visual Odometry algorithms, like the SVO\cite{SVOpaper}, one may calculate the object's relative position to the environment in real time.
\todo[inline]{Finish camera paragraph}

\todo[inline]{Add paragraphaph about 360-camera usage and differences}
\todo[inline]{Add motivation}



\section{Previous work}



\todo[inline]{Should contain work related to 360 cameras, modeling of 360 degree projections and what approaches there has been to simulating 360 footage}

\subsection*{citation testing}
\cite{Airsim_paper}, \cite{Sim4CV_paper}, \cite{Zhang2016BenefitOL}, \cite{endoscopypano}, \cite{GPSaccuracy}, \cite{SVOpaper}.