
\chapter{Results and Discussion}

\section{performance}

\subsection{compile time and runtime calculations}

\subsection{optimization}

\subsection{Discussion}
% This decision was made in order to separate the implementation into its own module, and thereby reducing the impact of changes to AirSim. As AirSim is still developed heavilly upon, it was seen as a way to reduce the amount trouble induced by changes to the core code of the plugin. This would also allow me to more often integrate the bug fixes and new implementations they made, without ruining my ownw work. The downside of this decision is that it removes possibility to use any features in Unreal Engine which is not implemented in the AirSim API. 


% \subsection{Project setup and programming environment}

% In this project I have forked both the EpicGames/Unreal Engine and the Microsoft/AirSim repository to my Github account, and made my own pivate repository for the project. This enable me to do specific changes to the source code of these projects, without the need to make pull requests to their original repositories. While AirSim is openly available, the source code of Unreal Engine is only available after being registered as a developer through their website. One requirement for getting the source code is that it is not distributed outside of this licencing. For this reason neither the fork of Unreal Engine, nor my own project repository are publicly available.

% Since one of the main goals of this project is the interfacing to ROS, the main OS for development will be Linux, specifically Ubuntu 18.04. However, since Unreal Engine is deeply integrated with Visual Studios, I will use Visual Studios on Windows as the main debugging platform, for everything except ROS. To build on Linux, I will use CMake, as they have done with AirSim. Both Unreal Engine and AirSim is built with the clang compiler, using libc++ as the standard library. However, the default ROS install uses libstdc++ as their standard library. This caused linking problems when combining ROS and AirSim. I therefore had to make some changes to the build and CMake scripts of AirSim in order to build with the GNU compiler g++, using libstdc++ as the standard library.

% \todo[inline]{Do I need to describe why this change is important?}

\cleardoublepage